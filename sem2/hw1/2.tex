\documentclass[a4paper,12pt]{article}
\usepackage[utf8x]{inputenc}
\usepackage[english,russian]{babel}
\usepackage{amsmath}
\usepackage{amsbsy}
\usepackage{listings}
\begin{document}

\begin{enumerate}
\setcounter{enumi}{3}
\item Идём по массиву с конца в начало и поддерживаем две вещи - минимум на пройденном нами суффиксе и сам ответ - индексы $i$ и $j$ такие, что $i < j$ и $a_i - a_j$ максимально. Обрабатывая новый элемент массива либо обновляем минимум, либо обновляем ответ, если разность его и текущего минимума больше, чем в текущем ответе. (Если я правильно понял задачу, а то формулировка не очень.)

\item Памагите, не понятно

\item Не принимая во внимание классическую задачу про прибавление на отрезке, можем считать, что все последовательности начинаются с нуля. Применим схожую идею и заведём ещё два массива, которые помогут поддерживать текущий шаг "суммарной" прогрессии, а также значение, аккумулирующее добавляемую разницу на каждом шаге. Т.к. на словах объяснить трудно, прилагаю код.

\item Заведём массив длины n и подсчитаем количество каждого значения в интервале $[1; n]$, далее идём по этому массиву пока сумма префикса не превышает $k$, значение, на котором сумма первышается - $k$ая статистика. По сути, осуществим сортировку подсчётом.

\item Ключевая идея - если на концах отрезка разные значения, то в нём есть соседи с различными значениями. Применяем бинарные поиск поддерживая на концах отрезка различные значения.

\item Применим бинарный поиск по ответу, искать будем в отрезке $[0; max({a_i})]$ т.к. ответ, если он существует, не может быть более $a_i$ ведь при $S=max({a_i})$ обезьянка съедает каждую кучку бананов за минимальное время - одну минуту. Валидность ответа будем проверять непосредственно подсчётом необходимо количества минут - за линию.


\end{enumerate}

\end{document}